%%-----------------------------------------------------------------------
% Loading the packages and classes
%%-----------------------------------------------------------------------

\documentclass[12pt,oneside,a4paper,final]{book}%
\usepackage{a4}%                        %% Verwendet mehr Platz auf einer A4 Seite als Option a4paper in book
\usepackage[german, ngerman, english]{babel}%    %% Babel Sprachen
\usepackage[utf8]{inputenc}%          %% input encoder für umlaute usw.
\usepackage[T1]{fontenc}
\usepackage{amssymb}%                   %% AMS Symbole
\usepackage{amsmath}%                   %% AMS Math Funktionen
\usepackage{amsfonts}%
\usepackage[Sonny]{fncychap}%           %% Chapter Style
\usepackage[english,noprefix]{nomencl}%  %% Nomenclature (Symbolverzeichnis)
\usepackage{makeidx}%                   %% Index
\usepackage{graphicx}%                  %% Grafiken
\graphicspath{{img/}}
\DeclareGraphicsExtensions{.pdf,.jpeg,.png,.jpg}
\usepackage{psfrag}%                    %% Tex-Schriftarten und Formeln in EPS-Grafiken
\usepackage{color}
\usepackage{nicefrac}
\usepackage{ifthen}
\usepackage{fancyhdr}
\usepackage{subfigure}
\usepackage[xindy, acronym, toc]{glossaries}
\usepackage{cite}
\usepackage{listings}
\usepackage{color}
\usepackage[dvipsnames]{xcolor}
\usepackage{rotating}
\usepackage{todonotes}
\usepackage{csquotes}
\usepackage{framed}
\usepackage{placeins}
\usepackage{afterpage}
\usepackage{eurosym}
\usepackage{acronym}


%Define Acronyms like
\newacronym{tgm}{TGM}{Technologisches Gewerbemuseum}
% use on every place in your document \gls{mas} for TGM or - for plural - use \glspl for TGMs
% at the first usage of this, the acronym will be introduced, everywhere else it will only be the in the short form: ``Technologisches Gewerbemuseum (TGM)''
% TIPP: USE THIS FOR EVERY NAME/SOFTWARE-TOOL/MAIN PART OF YOUR WORK, like JAVA, - so that, e.g. JAVA is not written Java everywhere else in your thesis.
\newacronym{gis}{GIS}{Gotti is supa!}
%%-----------------------------------------------------------------------
% Using the Hyperref-Package for PDF-Online Version
%%-----------------------------------------------------------------------


\def\usehyperref{1}

\ifnum\usehyperref=1
\usepackage[pdftex=true,
  pdftitle={Diplomarbeit1},
  pdfauthor={AUTOR},
  bookmarksopen,
  colorlinks,
  citecolor=black,
  linkcolor=black,
  breaklinks,
  urlcolor=black
]{hyperref}%
\fi



%%-----------------------------------------------------------------------
% Rearranging Nomenclature
%%-----------------------------------------------------------------------

\renewcommand{\nomname}{Abbildungsverzeichnis}
%\renewcommand{\nompreamble}{The following list only contains symbols
%  that are used continuously throughout the text. Local symbols are
%  not listed.}
\renewcommand{\nomgroup}[1]{
 \ifthenelse{\equal{#1}{A}}{\item[\textbf{General symbols}\bigskip]}{
 \ifthenelse{\equal{#1}{B}}{\item[\bigskip\bigskip\textbf{Chapter 2}\bigskip]}{
 \ifthenelse{\equal{#1}{C}}{\item[\bigskip\bigskip\textbf{Chapter 3}\bigskip]}{
 \ifthenelse{\equal{#1}{D}}{\item[\bigskip\bigskip\textbf{Chapter 4}\bigskip]}{
 \ifthenelse{\equal{#1}{E}}{\item[\bigskip\bigskip\textbf{Chapter 5}\bigskip]}{
 \ifthenelse{\equal{#1}{F}}{\item[\bigskip\bigskip\textbf{Appendix}\bigskip]}{
 }}}}}}}
\makenomenclature



%%-----------------------------------------------------------------------
% Makes Bibliography available in Winedt
%%-----------------------------------------------------------------------

%GATHER{bib_kiefer.bib}


%%-----------------------------------------------------------------------
% Definition of possible environments
%%-----------------------------------------------------------------------

\newtheorem{theorem}{Theorem}[chapter]
\newtheorem{acknowledgement}{Acknowledgement}[chapter]
\newtheorem{algorithm}{Algorithm}[chapter]
\newtheorem{axiom}{Axiom}[chapter]
\newtheorem{case}{Case}[chapter]
\newtheorem{claim}{Claim}[chapter]
\newtheorem{conclusion}{Conclusion}[chapter]
\newtheorem{condition}{Condition}[chapter]
\newtheorem{conjecture}{Conjecture}[chapter]
\newtheorem{corollary}{Corollary}[chapter]
\newtheorem{criterion}{Criterion}[chapter]
\newtheorem{definition}{Definition}[chapter]
\newtheorem{example}{Example}[chapter]
\newtheorem{exercise}{Exercise}[chapter]
\newtheorem{lemma}{Lemma}[chapter]
\newtheorem{notation}{Notation}[chapter]
\newtheorem{problem}{Problem}[chapter]
\newtheorem{proposition}{Proposition}[chapter]
\newtheorem{remark}{Remark}[chapter]
\newtheorem{solution}{Solution}[chapter]
\newtheorem{summary}{Summary}[chapter]
\newenvironment{proof}[1][Proof]{\noindent\textbf{#1.} }{\ \rule{0.5em}{0.5em}}

\renewcommand{\chaptermark}[1]{\markboth{\thechapter.\ #1}{}}
\renewcommand{\sectionmark}[1]{\markright{\thesection.\ #1}}




%%-----------------------------------------------------------------------
% Marking of overfull boxes and increasing of tolerances
%%-----------------------------------------------------------------------

% Für die Final-Version die nächste Zeile auskommentieren um schawarze Balken (TU-Logo im Titelblatt) zu ignorieren!
\overfullrule=10pt%                     %% Markiert überfüllte Boxen. z.b. hbox overfull (Evtl. nicht im pdf sichtbar!!!!)
\hfuzz=1pt%                             %% Toleranz bei hbox overfull erhöht 1pt entspr. ca. 1/3 mm


%%-----------------------------------------------------------------------
% Counter
%%-----------------------------------------------------------------------


\setcounter{secnumdepth}{3}%
\setcounter{tocdepth}{3}%



% Clear Header Style on the Last Empty Odd pages
\makeatletter
\def\cleardoublepage{\clearpage\if@twoside \ifodd\c@page\else%
    \hbox{}%
    \thispagestyle{empty}%              % Empty header styles
    \newpage%
    \if@twocolumn\hbox{}\newpage\fi\fi\fi}
\makeatother

%%-----------------------------------------------------------------------
% Avoid indents
%%-----------------------------------------------------------------------

\setlength{\parindent}{0pt}

%%-----------------------------------------------------------------------
%% Hyphenation for german abstract TODO
%%-----------------------------------------------------------------------

\hyphenation{Fa-mi-lie
             Ski-bil-dung
             Ar-beits-wal-ze
             neg-lec-ted
             se-par-ate
             di-men-sio-nal
             her-r\"uhren
             N\"a-herungs-l\"os-ungen
             wissen-schaft-licher
             Regelungs-technik
             re-con-fi-gur-abili-ty
             manage-ment
             manu-facturing
             not-wendigen
             Steu-er-ung
             }


%%-----------------------------------------------------------------------
% Colored grafix
% 1 = color
% 0 = grey
%%-----------------------------------------------------------------------


\def\colorsw{1}

%%-----------------------------------------------------------------------
% Additional remarks
% 1 = with remarks
% 0 = without remarks
%%-----------------------------------------------------------------------


\def\addnotes{0}

%%-----------------------------------------------------------------------
% Listing styles
%%-----------------------------------------------------------------------


\definecolor{Maroon}{rgb}{0.5,0,0}

\lstdefinestyle{ST}
{
	language=C,
	frame=trbl,
	%numbers=left, % Nummerierung
	%numberstyle=\tiny, % kleine Zeilennummern
	breaklines=true,
	showstringspaces=false,
	breakatwhitespace=true,
	escapeinside={(*@}{@*)},
	commentstyle=\color{ForestGreen},
	morekeywords={BOOL, VAR, END_VAR, LD, CR, ANDN, ST, AND, NOT, OR, INITIAL_STEP, END_STEP, TRANSITION, FROM, END_TRANSITION, STEP, TO, IF, ELSIF, END_IF, THEN, CASE, END_CASE},
	keywordstyle=\color{Blue},
	stringstyle=\color{Maroon},
  basicstyle=\ttfamily\footnotesize,
	backgroundcolor=\color{Yellow!5}
}

\lstdefinestyle{XML}
{
	language=xml,
	frame=trbl,
	tabsize=2,
  basicstyle=\ttfamily\footnotesize,
  morestring=[s]{"}{"},
  morecomment=[s]{<!--}{-->},
  commentstyle=\color{ForestGreen},
  moredelim=[s][\color{Red}]{\ }{=},
  stringstyle=\color{Blue},
	tagstyle=\color{Maroon},
	backgroundcolor=\color{Yellow!5}
}

\lstdefinestyle{csharp}
{
	language=[Sharp]C,
	frame=trbl,
	%numbers=left, % Nummerierung
	%numberstyle=\tiny, % kleine Zeilennummern
	breaklines=true,
	showstringspaces=false,
	breakatwhitespace=true,
	escapeinside={(*@}{@*)},
	commentstyle=\color{ForestGreen},
	morekeywords={partial, var, value, get, set},
	keywordstyle=\color{Blue},
	stringstyle=\color{Maroon},
  basicstyle=\ttfamily\footnotesize,
	backgroundcolor=\color{Yellow!5}
}

\lstdefinestyle{VBA}
{
	language=[Visual]Basic,
	frame=trbl,
	breaklines=true,
	showstringspaces=false,
	breakatwhitespace=true,
	commentstyle=\color{ForestGreen},
	keywordstyle=\color{Blue},
	stringstyle=\color{Maroon},
  basicstyle=\ttfamily\footnotesize,
	backgroundcolor=\color{Yellow!5}
}


%%-----------------------------------------------------------------------
% Define month
%%-----------------------------------------------------------------------

\def\monthdis{April 2015}

\makeglossaries

%%-----------------------------------------------------------------------
% Document
%%-----------------------------------------------------------------------


\begin{document}%
\selectlanguage{german}%
\renewcommand{\indexname}{Index}%
\topmargin15.0mm


\def\tpdefault{{\sf \center \vspace*{-4cm}
%\begin{center}
%\hspace*{-1.3cm}
%\rule{17cm}{0.02cm}
%\end{center}


\begin{figure}[h]
\begin{flushright}	
		\includegraphics[width=0.3\textwidth]{graphics/title/tgmlogo2.png}
	\label{fig:tgmlogo}
\end{flushright}
\end{figure}


\vspace{2cm}


{\Large %\bf 
DIPLOMARBEIT\\ \vspace{0.7cm}}
 {\LARGE \sloppy
{\bf \sf  \textbf{Batch\_it}
\\}}
%
%
\vspace*{2cm}
{\normalsize Ausgeführt im Zuge der Reife und Diplomprüfung\\
Ausbildungszweig Systemtechnik\\ %unzutreffendes streichen
  \vspace{1.5cm}
  \normalsize unter der Leitung von\\
  \large Dipl.-Ing.~(FH)\ Mag.\ Dr.techn.\ Gottfried Koppensteiner \\
  \normalsize Abteilung für
  Informationstechnologie\\
  \vspace{1.5cm}
  eingereicht am  Technologischen Gewerbemuseum Wien\\
  H\"ohere Technische Lehr- und Versuchsanstalt\\
  Wexstrasse 19-23, A-1200 Wien\\
  }}}


\begin{titlepage}
	\tpdefault
	{\sf \center \vspace{1.0cm}
	\normalsize von\\
	\large 
	Paul Adeyemi, 5AHITT\\
	Jakob Saxinger, 5AHITT\\
	Nikolaus Schrack, 5AHITT\\
	Philipp Schwarzkopf, 5AHITT\\
	\vspace {2 cm}
	\bf \sf {Wien, im \monthdis} \\
		%	\vspace{2cm}
	%	\rule{\textwidth}{0.01cm}
	
	}



	\end{titlepage}

\begin{titlepage}
	{\color{white}.}
	\bigskip
	\vspace{14cm}
	%\vfill%
	\noindent%

	Abteilungsvorstand:\hfill Dipl.-Ing.~(FH)\ Mag.\ Dr.techn.\ Gottfried Koppensteiner\\
	\bigskip
	\bigskip

	Tag der Reifeprüfung:\hfill 16/17.06.2015\\
	\bigskip
	\bigskip

	Prüfungsvorsitzender:\hfill
	LSI Mag. Bernd Steiner\\
	\smallskip

	Erster Gutachter:\hfill Dipl.-Ing.(FH)\ Mag. \ Dr.techn.\ Gottfried Koppensteiner\\
	\smallskip

	Zweiter Gutachter:\hfill 	Dipl.-Ing.\ Wilfried\ Lepuschitz\\
		\smallskip
\end{titlepage}

\frontmatter%   %% front matter will be numbered in small Roman letters

%%-----------------------------------------------------------------------


%TCIDATA{OutputFilter=latex2.dll}
%TCIDATA{Version=5.00.0.2552}
%TCIDATA{LaTeXparent=0,0,Dissertation_SW.tex}


\chapter*{Vorwort}

Diese Arbeit wurde im Jahr 2015 im Zuge unserer Ausbildung in der Abteilung für Informationstechnologie am \gls{tgm}, HTBLVA Wien 20, durchgeführt. 


\bigskip

Zunächst möchten wir uns an dieser Stelle bei all jenen bedanken, die uns während der Erstellung dieser Diplomarbeit unterstützt und motiviert haben. \\\\
Besonderer Dank gilt unseren Projektbetreuern Dipl.-Ing. (FH) Mag. Dr.techn. Gottfried Koppensteiner und Herrn Dipl.-Ing. Wilfried Lepuschitz für die geduldige Begleitung und konstruktive Kritik sowie Herrn Alvaro Lobato-Jimenez für sein hilfreiches Engagement.
Zusätzlich bedanken wir uns beim Forschungsinstitut PRIA, Practical Robotics Institute Austria, für die großzügige finanzielle Unterstützung.\\\\
Nicht zuletzt gebührt unseren Eltern Dank, die uns nicht nur finanziell, sondern auch moralisch immer zur Seite gestanden sind und uns den Rücken gestärkt haben.
\bigskip
\bigskip
\bigskip



Wien, im \monthdis \\ 
\hfill Paul Adeyemi, Jakob Saxinger, Nikolaus Schrack, Philipp Schwarzkopf \vfill
%
\chapter*{Abstract}

The manufacturing technology has to meet the growing demands of the 21st century. The configurable mass production, which makes it possible to respond to individual customer requirements, brings great challenges. Manufacturing systemas are requiered to support the capability of made to order instead of made to stock. The order-related production requires a short processing time by remaining high quality and little or no extra cost compared to the conventional serial and mass production. \\\\
Nowadays manufacturing systems are often not able to cope with this requirement due to their rigid and therefore inflexible structure. Great efforts are needed to put the system in a new composition when parts of the manufacturing systems are removed or changed. Especially for the control units of manufacturing plant a lot of the program code must be re-implemented.\\\\
The aim of the thesis is to automate this implementation step on a batch process plant. For this, a laboratory facility has been designed and constructed. In addition, a control and visualization application using the zenon software was created. In the final step the model-based development of the control system could be implemented.\\\\
The PLC code was generated with an ontology-based information model, to represent the production system, and an activity diagram, to define the procedures. The ontology and the activity diagram were exported as an \ac{XML} file and with the zenon Wizard a program was written that creates parts of the control system automatically.%
\selectlanguage{ngerman}%
\chapter*{Kurzfassung}

Die Anlagen- und Verfahrenstechnik muss den wachsenden Anforderungen des 21. Jahrhunderts gerecht werden. Die konfigurierbare Massenproduktion, die es erlaubt, auf individuelle Kundenwünsche einzugehen, bringt große Herausforderungen mit sich. Produktionssysteme müssen in der Lage sein Waren auf Bestellung anstatt von Serienfertigung zu erzeugen. Die auftragsbezogene Einzelproduktion erfordert eine kurze Durchlaufzeit bei hoher Qualität und nur geringe oder keine Mehrkosten gegenüber der herkömmlichen Serien- und Massenproduktion. \\\\
Heutzutage sind Produktionssysteme wegen ihres starren Aufbaus diesen Anforderungen oft nicht gewachsen. Bei Änderungen wie dem Hinzufügen oder Entfernen von Teilen sind große Aufwände nötig, um das System in neuer Zusammensetzung in Betrieb zu setzen. Besonders für die Steuereinheiten von Produktionsanlagen muss ein Großteil des Programm-Codes neu implementiert werden.\\\\
Das Ziel der Diplomarbeit ist es, diesen Implementierungsschritt auf einer Chargenprozessanlage zu automatisieren. Dazu ist eine Laboranlage konzeptioniert und aufgebaut worden. Darüber hinaus wurde eine Steuerungsapplikation mit Visualisierung mithilfe der Software zenon erstellt. Im letzten Schritt konnte die modellbasierte Entwicklung der Steuerung umgesetzt werden. 
\\\\
Der \ac{SPS} Code wurde einerseits auf einem Ontologie-basierten Informationsmodell, um das Produktionssystem abzubilden, und andererseits auf einem Aktivitätsdiagramm, um die Prozeduren zu definieren, generiert. Die Ontologie und das Aktivitätsdiagramm sind als \ac{XML}-File exportiert worden und mittels des zenon Wizard wurde ein Programm geschrieben, dass daraus Teile der Steuerung automatisch erstellt.
%Es beginnt mit einer Produktionsanlage, wobei es sich um eine Chargenprozessanlage handelt, die in einer Ontologie abgebildet wird. Die Ontologie und das Diagramm der Prozeduren beinhaltet die nötigen Daten, die gebraucht werden, um im weiteren Schritt den Code für die SPS zu generieren. Dieser Code ist nach dem IEC 61512 Standard genormt. Nachdem der Code in die SPS eingespielt wird, kann die Chargenprozessanlage gesteuert werden. 
%
\selectlanguage{german}%


%%-----------------------------------------------------------------------
% Define Header for Content chapter
%%-----------------------------------------------------------------------


\makeatletter
\def\tableofcontents{\chapter*{\contentsname\@mkboth{\contentsname}{\contentsname}}
  \@starttoc{toc}}
\makeatother

\clearpage%
\tableofcontents
\clearpage
\listoffigures
\clearpage
\lstlistoflistings %\listoftables
\clearpage 
\chapter{Akronyme}
\begin{acronym}
  \acro{SPS}{Speicherprogrammierbare Steuerung}
  \acro{KOP}{Kontaktplan}
  \acro{FBS}{Funktionsbausteinsprache}
  \acro{AWL}{Anweisungsliste}
  \acro{ST}{Strukturierter Text}
  \acro{VBA}{Visual Basic for Applications}
  \acro{VSTA}{Visual Studio Tools for Applications}  
  \acro{HMI}{Human Machine Interface}
  \acro{XML}{eXtensible Markup Language}
  \acro{XMI}{XML Metadata Interchange}
  \acro{SCADA}{Supervisory Control and Data Acquisition}
  \acro{UML}{Unified Modeling Language}
  \acro{OWL}{Web Ontology Language}
  \acro{RDF}{Resource Description Framework}
  \acro{PRIA}{Practical Robotics Institute Austria}
  \acro{BatMAS}{Batch Process Automation with an Ontology-driven Multi-Agent System}
  \acro{DIN}{Deutsches Institut für Normung}
  \acro{EN}{Europäische Norm}
  \acro{IEC}{International Electrotechnical Commission}
  \acro{CPU}{Central Processing Unit}
  \acro{PS}{Power Supply}
\end{acronym}

\clearpage
\markboth{Contents}{Contents}


%\addcontentsline{toc}{chapter}{\numberline{}\listfigurename}%
%\listoffigures
%\listoftables%
%\addcontentsline{toc}{chapter}{\numberline{}\listtablename}%
\clearpage%


\include{tex/nomenclature}
\markboth{\nomname}{\nomname}%
\addcontentsline{toc}{chapter}{\numberline{}\nomname}%
\printnomenclature


\mainmatter%   %% main part will be numbered in Arabic letter


% include chapters
% Chapter1
\chapter{Einleitung} \label{chapter:introduction}

% Inhalt
% Motiviert zum Thema und führt zum Thema hin.
% Erklärt wie man löst  
% Hintergrund und Ausgangspunkt (Heutzutage statische Systeme .. Dynamik… ) Wir wollen Methoden zur Verbesserung der Anlagen ausprobieren 
% Aufgabenstellung
% Leitfaden durch die Arbeit 		

% \section{Background and Motivation} 
% \section{Objectives of this Thesis}
% Ziel dieser Abschlussarbeit
% \section{Methodology for the Developement}  
% Methoden zur Entwicklung 
% \section{Thesis Outline}
% Abschlussarbeit Gliederung


Heutzutage sind Produktionsanlagen so konstruiert, dass die Reihenfolge der einzelnen Fertigungszellen fest miteinander verkettet ist. Es herrscht eine unbewegliche, statische Folge der Stationen in dem jedes Anlagemodul autark arbeitet. Wenn es bei einem Teil zu einer Störung kommt, steht der ganze Produktionsfluss ausnahmslos. Zusätzlich ist die Flexibilität der Einsatzmöglichkeit einer Produktionsanlage eingeschränkt. Neue Module in die Verkettung hinzuzufügen erfordert enorme Umbauten und Kosten. \\\\
Weiters kommt dazu, dass der Code auf der SPS neu programmiert werden muss, wenn ein Teil der Anlage anders verwendet werden soll. Dies hat einen Stillstand der Fabrik zur Folge. Allgemein sind Produktionsanlagen sehr statisch gestaltet und können nur mit viel Aufwand geändert und angepasst werden. Bei Störungen kann nur schlecht reagiert werden was wiederum fatale Ausfälle in der Produktion nach sich zieht.

\section{Hintergrund und Ausgangspunkt}


\section{Aufgabenstellung}

\section{Leitfaden durch die Arbeit}
%
% Chapter2

\chapter{State of the Art} \label{chapter:stateoftheart}

\begin{quotation}
``The improvement of \textbf{understanding} is for two ends: first, our own increase of knowledge; secondly, \textit{to enable us to deliver} that knowledge to others.``
\begin{flushright}
(John Locke)
\end{flushright}
\end{quotation}


Beste Code ever.

\lstset{language=java, basicstyle=\scriptsize, keywordstyle=\color{Purple}\bfseries, commentstyle=\color{OliveGreen}, showstringspaces=false, stringstyle=\ttfamily, breaklines=true, numbers=left, numberstyle=\tiny, frame=single, caption=[JADE-Example: HelloWorldAgent.]{JADE-Example: HelloWorldAgent~\cite{bellifemine_book_2007}}}

\section{dsadsfdsads}
sdfds

\subsection{asdhfjkdsahfkasdf}


Der Code ist so gut wie die Referenz~\cite{bellifemine_book_2007}\footnote{huhuhuhuhuhuh}. 

\begin{lstlisting} 
import jade.core.Agent;
public class HelloWorldAgent extends Agent 
{
protected void setup() 
 {
 // Printout a welcome message
 System.out.println(``Hello World. I'm an agent!'');
 }
}
\end{lstlisting}

\lstset{language=c, basicstyle=\scriptsize, keywordstyle=\color{Purple}\bfseries, commentstyle=\color{OliveGreen}, showstringspaces=false, stringstyle=\ttfamily, breaklines=true, numbers=left, numberstyle=\tiny, frame=single, caption=[JADE-Example: HelloWorldAgent.]{JADE-Example: HelloWorldAgent~\cite{bellifemine_book_2007}}}

\begin{lstlisting} 
import jade.core.Agent;
public class HelloWorldAgent extends Agent 
{
protected void setup() 
 {
 // Printout a welcome message
 System.out.println(``Hello World. I'm an agent!'');
 }
}
\end{lstlisting}%
% Chapter3

\chapter{Konzept/Architektur} \label{chapter:architecture}


\section{Einleitung}
\section{Modell der Anlage}
\section{Diagramm der Phasen}
\section{Codegenerierung}


TODO%
% Chapter4
\chapter{Implementierung} \label{chapter:thevetestcase}
\section{Modellierung der Anlage}
\section{Aufbau der Anlage}
\section{Phasen und Rezepte}
\section{HMI}
\section{Ontology}
\section{Aktivitätsdiagramm}
\section{Codegenerierung}

TODO%
% Chapter5

\chapter{Evaluation} \label{chapter:evaluation}




TODO%


\chapter{Zusammenfassung und Ausblick} \label{chapter:conclusion}
\section{Zusammenfassung}
In dem Projekt Batch\_it wurden 
\section{Ausblick}%


\addcontentsline{toc}{chapter}{Glossar} 
\printglossary[type=\acronymtype]
\glsaddall

\printglossary[type=\acronymtype]
%\printglossary[type=\acronymtype,style=listwithwidth]


%% include appendix
\begin{appendix}
\chapter{Appendix}
	%\label{appendix}
	
	\begin{figure}[h!]
  		\centering
      	\includegraphics[width=1\textwidth]{graphics/implementation/RI_Impl_farblos}
  		\caption{Finales Rohr- \& Instrumentenfließschema}
	\end{figure}	
	
	\begin{figure}[h!]
  		\centering
      	\includegraphics[width=1\textwidth]{graphics/implementation/FinalerAufbau2}
  		\caption{Finaler Aufbau der Anlage}
	\end{figure}	%
\end{appendix}

% Use of the sorted IEEE style, with changes:
% "dashification" was disabled


\cleardoublepage

% IEEE Style
\bibliographystyle{sty/IEEEtranS}

% GATHER
%\input "bib_file.bib"
\phantomsection{}
\addcontentsline{toc}{chapter}{\bibname}
\pagestyle{myheadings}\markboth{\bibname}{\bibname}
\bibliography{tex/bib_file}

%%% generate index
%\clearpage%
%\markboth{\indexname}{\indexname}%
%\printindex%
%\addcontentsline{toc}{chapter}{\numberline{}\indexname}%

%% include affidavit
\thispagestyle{empty}
\vspace*{2cm}
\begin{center}
{\bf \sf \huge Erkl{\"a}rung}
\end{center}
{\sf \vspace{1cm} Hiermit erkl{\"a}ren wir, dass die vorliegende
Arbeit ohne unzul{\"a}ssige Hilfe Dritter und ohne Benutzung
anderer als der angegebenen Hilfsmittel angefertigt wurde. Die aus
anderen Quellen oder indirekt übernommenen Daten und Konzepte sind
unter Angabe der Quelle gekennzeichnet.

Die Arbeit wurde bisher weder im In- noch im Ausland in gleicher
oder in {\"a}hnlicher Form in anderen Pr{\"u}fungsverfahren
vorgelegt.
\\[1.5cm]
Wien, im \monthdis
\\[2cm]
Paul Adeyemi
\\[2cm]
Jakob Saxinger
\\[2cm]
Nikolaus Schrack
\\[2cm]
Philipp Schwarzkopf
}%end sf
%



\end{document}
%%%%%%%%%%%%%%%%%%%%%%%%%%%%%%%%%%%%%%%%%%%%%%%%%%%%%%%%%%%%%%%%%%%%%%%%%%%%%%%%%%%%%%%%%%%%%%%%%%%
%%%%%%%%%%%%%%%%%%%%%%%%%%%%%%%%%%%%%%%%%% End Dokument %%%%%%%%%%%%%%%%%%%%%%%%%%%%%%%%%%%%%%%%%%%
%%%%%%%%%%%%%%%%%%%%%%%%%%%%%%%%%%%%%%%%%%%%%%%%%%%%%%%%%%%%%%%%%%%%%%%%%%%%%%%%%%%%%%%%%%%%%%%%%%%
