% Chapter3

\chapter{Konzept/Architektur} \label{chapter:architecture}
Die Abfolge der Arbeitsschritte für die modellbasierte Entwicklung der Steuerung einer Chargenprozessanlage sieht wie folgt aus. 
\begin{figure}[h!]
		\centering
		\includegraphics[width=0.37\textwidth]{graphics/konzept/konzept_allblack.pdf}
		\caption{Konzept für die modellbasierte Entwicklung der Steuerung}
		%\label{fig:MegaProxyProblem}
\end{figure} \\
Es beginnt mit einer Produktionsanlage, wobei es sich um eine Chargenprozessanlage handelt, die in einem Modell abgebildet wird. Dieses Modell beinhaltet die nötigen Daten, die gebraucht werden, um im weiteren Schritt den Code für die SPS zu generieren. Dieser Code ist nach dem IEC 61512 Standard genormt. Nachdem der Code in die SPS eingespielt wird, kann die Chargenprozessanlage gesteuert werden. 
In diesem Kapitel wird jeder Schritt, der in dem Kreislauf von Abbildung 3.1 enthalten ist, genauer elaboriert und in Perspektive zu dem entwickelten Konzept gebracht.
\section{Produktionsanlage}
\section{Modell}
Die Informationen für die Codegenerierung benötigen einerseits einen Ontologie, zur Speicherung der Informationen der Produktionsanlagen,  und andererseits ein Aktivitätsdiagramm. In diesem werden die Wege (Phasen) dargestellt die später möglich sein sollen.
Die beiden sind von einander Abhängig, denn die Namen, der von in der Ontologie erstellen Identitäten (Datensätze), werden den Namen des Aktivitätsdiagramms zugeordnet.
Mithilfe der Ontologie und des Aktivitätsdiagramms können die Informationen bereitgestellt werden, die für die Codegenerierung nötige sind.  

\subsection{Modell der Produktionsanlage}
Hierbei handelt es sich um die Speicherung der Daten der Produktionsanlage in einer Ontologie. Je genauer diese im Modell abgebildet ist, desto mehr kann automatisch generiert werden. Wenn zum Beispiel maximum und minimum Werte einfließen, können diese berücksichtigt werden.
\begin{figure}[hbt!]
 \centering
  \includegraphics[width=1\textwidth]{graphics/stateoftheart/Ontology_Aufbau}
  \caption{Aufbau einer Ontologie}
\end{figure}
\subsection{Diagramm der Wege}
Das Modell der Wege wird in einem Activity Diagram nach der UML Norm erstellt. Dieses enthält die Phasen die später Codegeneriert werden und im weiteren Schritt in der Chargenprozessanlage verwendbar sein sollen.
Jede Phase wird durch einen Start- und Endpunkt definiert. Dazwischen werden die Namen der Stadtionen geschrieben und diese mit Pfeilen verbunden. Das Diagramm für zwei Phasen würde ca. so aussehen: 
\begin{figure}[h!]
		\centering
		\includegraphics[width=0.7\textwidth]{graphics/konzept/UML_Activity.png}
		\caption{Abbildung der Wege mittels Aktivitätsdiagramm}
		%\label{fig:MegaProxyProblem}
\end{figure}
\newpage
\section{SPS Code}
\subsection{Phasen}
\subsection{Rezepte}
\section{Codegenerierung}
\section{Visualisierung der Produktionsanlage}

