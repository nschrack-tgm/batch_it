\chapter*{Kurzfassung}

%Der Bereich der industriellen Fertigung muss im 21. Jahrhundert wachsenden Anforderungen gerecht werden. Die Dynamik des Marktes erfordert eine konfigurierbare Massenproduktion, die es erlaubt auf individuelle Kundenwünsche einzugehen. Es wird von Produktionssystemen erwartet, dass sie „made-to-order“ anstatt von „made-to-stock“ Waren erzeugen können. Besonders die auftragsbe- zogene Einzelproduktion bringt große Herausforderung mit sich. Dabei wird eine kurze Durchlaufzeit bei hoher Qualität ebenso erwartet, sowie keine oder nur ge- ringe Mehrkosten gegenüber der herkömmlichen Serien- und Massenproduktion. Zusätzlich verringern sich laufend die Lebenszyklen von Waren, die wiederum eines flexibleren Produktionssystems bedürfen. Diese sind heutzutage allerdings wegen ihres starren Aufbaus oft nicht im Stande diese Anforderungen zu erfüllen. [4]
%Diese Anpassungsfähigkeit fordert ein schnelles Umstellen von Produktionssy- stemen. Hingegen ist es vielfach so, dass bei Änderungen wie dem Hinzufügen oder Entfernen von Teilen großer Aufwand nötig ist, um das System in neuer Zusammensetzung in Betrieb zu setzen. Besonders für die Steuereinheiten von Produktionsanlagen, die sog. speicherprogrammierbare Steuerung (\ac{SPS}), muss ein Großteil des Programm-Codes neu implementiert werden. [5] Das Diplomprojekt hat sich zur Aufgabe gemacht, diesen Implementierungsschritt zu untersuchen und zu klären, in welchem Ausmaß und in welcher Art und Weise sich dieser Schritt anhand einer Implementierung auf einer Laboranlage automatisieren lässt. Diese Problematik der Automatisierung wird in den nächsten Abschnitten weiter beleuchtet. Dabei wird auf Motivation und Hintergrund der Arbeit eingegangen und der Stand der bereits realisierten Projekte erläutert. Ein Überblick über die Aufgabenstellung sowie ein Leitfaden durch die Arbeit schließen dieses Kapitel

%Die Idee ist, diesen Implementierungsschritt auf einer Laboranlage zu automatisieren. Der \ac{SPS} Code soll auf einem Ontologie-basierten Informationsmodell, um das Konzept von Produktionssystemen zu beschreiben, sowie einem Aktivitätsdiagramm, um die Prozeduren zu definieren, generiert werden. Wenn es nun zu Änderungen der Produktionsanlage kommt, können diese im Modell angepasst werden und  durch eine Codegenerierung der nötige Anpassungsschritt der Implementierung automatisiert werden. Zusätzlich soll die Integrierbarkeit des Codes in einen Steuerungsapplikation mit Visualisiserung mittels der Software „zenon“ von COPA-DATA getestet werden.\\\\

%Das Projekt Batch\_it hat sich zum Ziel gesetzt eine Laboranlage zu konzeptioniert und aufzubauen. Als weiteres sollte für diese eine Steuerungsapplikation mit Visualisierung mithilfe der Software zenon erstellt werden. Im letzten Schritt sollte die modellbasierte Entwicklung der Steuerung umgesetzt werden. Hierzu wurde eine Ontologie benötigt, welche die Laboranlage abbildet, und ein Aktivitätsdiagramm, welches die Prozeduren definiert. Anhand dieser Daten sollte eine Codegenerierung für die Steuerung der Anlage implementiert werden. \\\\
Die Anlagen- und Verfahrenstechnik muss den wachsenden Anforderungen des 21. Jahrhunderts gerecht werden. Die konfigurierbare Massenproduktion, die es erlaubt, auf individuelle Kundenwünsche einzugehen, bringt große Herausforderungen mit sich. Produktionssysteme müssen in der Lage sein auf Bestellung anstatt von Serienfertigung Waren erzeugen. Die Auftragsbezogene Einzelproduktion erfordert eine kurze Durchlaufzeit bei hoher Qualität und nur geringe oder keine Mehrkosten gegenüber der herkömmlichen Serien- und Massenproduktion. \\\\
Heutzutage sind Produktionssysteme wegen ihres starren Aufbaus diesen Anforderungen oft nicht gewachsen. Bei Änderungen wie dem Hinzufügen oder Entfernen von Teilen sind große Aufwände nötig, um das System in neuer Zusammensetzung in Betrieb zu setzen. Besonders für die Steuereinheiten von Produktionsanlagen muss ein Großteil des Programm-Codes neu implementiert werden.\\\\
Das Ziel der Diplomarbeit ist es, diesen Implementierungsschritt auf einer Chargenprozessanlage zu automatisieren. Dazu ist eine Laboranlage konzeptioniert und aufgebaut worden. Darüber hinaus wurde eine Steuerungsapplikation mit Visualisierung mithilfe der Software zenon erstellt. Im letzten Schritt konnte die modellbasierte Entwicklung der Steuerung umgesetzt werden. 
\\\\
Der \ac{SPS} Code wurde einerseits auf einem Ontologie-basierten Informationsmodell, um das Produktionssystem abzubilden, und andererseits auf einem Aktivitätsdiagramm, um die Prozeduren zu definieren, generiert. Die Ontologie und das Aktivitätsdiagramm sind als \ac{XML}-File exportiert worden und mittels des zenon Wizard wurde ein Programm geschrieben, dass daraus Teile der Steuerung automatisch erstellt.\\\\
%Es beginnt mit einer Produktionsanlage, wobei es sich um eine Chargenprozessanlage handelt, die in einer Ontologie abgebildet wird. Die Ontologie und das Diagramm der Prozeduren beinhaltet die nötigen Daten, die gebraucht werden, um im weiteren Schritt den Code für die SPS zu generieren. Dieser Code ist nach dem IEC 61512 Standard genormt. Nachdem der Code in die SPS eingespielt wird, kann die Chargenprozessanlage gesteuert werden. 
