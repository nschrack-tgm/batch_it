\chapter*{Kurzfassung}

Die Anlagen- und Verfahrenstechnik muss den wachsenden Anforderungen des 21. Jahrhunderts gerecht werden. Die konfigurierbare Massenproduktion, die es erlaubt, auf individuelle Kundenwünsche einzugehen, bringt große Herausforderungen mit sich. Produktionssysteme müssen in der Lage sein Waren auf Bestellung anstatt von Serienfertigung zu erzeugen. Die auftragsbezogene Einzelproduktion erfordert eine kurze Durchlaufzeit bei hoher Qualität und nur geringe oder keine Mehrkosten gegenüber der herkömmlichen Serien- und Massenproduktion. \\\\
Heutzutage sind Produktionssysteme wegen ihres starren Aufbaus diesen Anforderungen oft nicht gewachsen. Bei Änderungen wie dem Hinzufügen oder Entfernen von Teilen sind große Aufwände nötig, um das System in neuer Zusammensetzung in Betrieb zu setzen. Besonders für die Steuereinheiten von Produktionsanlagen muss ein Großteil des Programm-Codes neu implementiert werden.\\\\
Das Ziel der Diplomarbeit ist es, diesen Implementierungsschritt auf einer Chargenprozessanlage zu automatisieren. Dazu ist eine Laboranlage konzeptioniert und aufgebaut worden. Darüber hinaus wurde eine Steuerungsapplikation mit Visualisierung mithilfe der Software zenon erstellt. Im letzten Schritt konnte die modellbasierte Entwicklung der Steuerung umgesetzt werden. 
\\\\
Der \ac{SPS} Code wurde einerseits auf einem Ontologie-basierten Informationsmodell, um das Produktionssystem abzubilden, und andererseits auf einem Aktivitätsdiagramm, um die Prozeduren zu definieren, generiert. Die Ontologie und das Aktivitätsdiagramm sind als \ac{XML}-File exportiert worden und mittels des zenon Wizard wurde ein Programm geschrieben, dass daraus Teile der Steuerung automatisch erstellt.
%Es beginnt mit einer Produktionsanlage, wobei es sich um eine Chargenprozessanlage handelt, die in einer Ontologie abgebildet wird. Die Ontologie und das Diagramm der Prozeduren beinhaltet die nötigen Daten, die gebraucht werden, um im weiteren Schritt den Code für die SPS zu generieren. Dieser Code ist nach dem IEC 61512 Standard genormt. Nachdem der Code in die SPS eingespielt wird, kann die Chargenprozessanlage gesteuert werden. 
