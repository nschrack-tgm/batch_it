

\chapter{Zusammenfassung und Ausblick} \label{chapter:conclusion}
%was wurde in der Arbeit gemacht`? 
%Zusammenfassung was gemacht wurde, 1 Seite min
%was ist übrig geblieben ? -> Ausblick 
%Das wäre toll gewesen wenn man es lösen hätte können --> aus welchen gründen. 
%Wir haben nicht den wirtschaftlichen Aspekt 

%Das Diplomprojekt hat sich zur Aufgabe gemacht, diesen Implementierungsschritt zu untersuchen und zu klären, in welchem Ausmaß und in welcher Art und Weise sich dieser Schritt anhand einer Implementierung auf einer Laboranlage automatisieren lässt. 
Im Projekt Batch\_it wurde die Möglichkeit untersucht die Entwicklung der Steuerung einer Chargenprozessanlage mittels einem modellbasierten Ansatz zu automatisieren. Das Ziel war es zu klären, in welchem Ausmaß und in welcher Art und Weise sich dieser Schritt auf einer Laboranlage automatisieren lässt. Dazu wurde eine Chargenprozessanlage geplant und aufgebaut sowie eine Steuerungsapplikation mit Visualisierung mittels zenon implementiert. 
Mit der Abbildung der Anlage in einer Ontologie und dem Definieren der Prozeduren in einem Aktivitätsdiagramm konnten der Code für die Steuerung generiert werden. \\\\
Die vollkommene Automatisierung ist nicht gelungen, da der zenon Wizard nicht alle Funktionalitäten unterstützte. So müssen manuell die Prozeduren in zenon hinzugefügt werden und die generierten Variablen mit der Visualisierung verknüpft werden. Das automatisierte Erstellen von Rezepten ist nicht erfolgreich gewesen, da das Integrieren in Batch Control nicht unterstützt wird. Ob das Konzept der modellbasierten Entwicklung der Steuerung wirtschaftliche Vorteile bringt, ist nicht untersucht worden. \\\\
Das Diplomprojekt wurde im Rahmen des Vereins \ac{PRIA} in Beziehung mit dem Projekt \ac{BatMAS} durchgeführt. Dieses forscht an einem ontologiebasierten Informationsmodell um die Konzepte von Produktionssystemen zu beschreiben. Dabei sollen intelligente Softwarekomponenten (Agenten) eingesetzt werden, die Aufträge dynamische zuteilen um die Produktionsdauer zu verringern und damit den Durchsatz des Systems zu erhöhen. Batch\_it konnte den Grundstein für die Validierung und Evaluierung des vorgestellten Ansatzes legen und erste Ergebnisse liefern.\\\\
%Dies repräsentiert die Basis für die Handlungsfähigkeit von intelligenten Softwarekomponenten (Agenten), die für die dynamische Zuteilung von Aufträgen genutzt werden, um die Produktionsdauer zu reduzieren und damit den Durchsatz des Systems zu erhöhen. Die Validierung und Evaluierung des vorgestellten Ansatzes ist anhand einer Demoimplementierung auf einer Laboranlage geplant.
Weiterführend wäre die Integrierung eines Routingalgorithmus für die Wegfindung eine sinnvolle Erweiterung. Dadurch könnte das Definieren der Prozeduren im Aktivitätsdiagramm hinfällig werden und allein anhand der Ontologie alle möglichen Prozeduren automatisiert generiert werden. Die Information, die für diesen Ansatz benötigt werden, sind jetzt schon in der Ontologie abgebildet. Es ist mögliche jede Verbindung zwischen Elementen inklusive der Richtung auszulesen. Der Algorithmus könnte im Betrieb auf alle Prozeduren zurück greifen und die bestmögliche Strecke finden.
Somit wäre die Anlage im Stande die redundanten Wege auszunutzen, indem mehrere Rezepte, ohne sich in die Quere zu kommen, gleichzeitig laufen.

