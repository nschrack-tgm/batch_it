\chapter*{Abstract}

The manufacturing technology has to meet the growing demands of the 21st century. The configurable mass production, which makes it possible to respond to individual customer requirements, brings great challenges. Manufacturing systemas are requiered to support the capability of made to order instead of made to stock. The order-related production requires a short processing time by remaining high quality and little or no extra cost compared to the conventional serial and mass production. \\\\
Nowadays manufacturing systems are often not able to cope with this requirement due to their rigid and therefore inflexible structure. Great efforts are needed to put the system in a new composition when parts of the manufacturing systems are removed or changed. Especially for the control units of manufacturing plant a lot of the program code must be re-implemented.\\\\
The aim of the thesis is to automate this implementation step on a batch process plant. For this, a laboratory facility has been designed and constructed. In addition, a control and visualization application using the zenon software was created. In the final step the model-based development of the control system could be implemented.\\\\
The PLC code was generated with an ontology-based information model, to represent the production system, and an activity diagram, to define the procedures. The ontology and the activity diagram were exported as an \ac{XML} file and with the zenon Wizard a program was written that creates parts of the control system automatically.