% Chapter5

\chapter{Evaluation} \label{chapter:evaluation}

%Einleitung + Aufgabenstellung
%3 Absätze mit länge halbe - 1 1/2 Seiten 
%Fragen werden beantwortet
%warum hats funktioniert oder warum hat es nicht funktioniert-> Begründung


%Mit welchen Technologien ist die Implementierung von Code einer SPS einer Chargenprozessanlage automatisierbar?

%Wie genau muss die Chargenprozessanlage abgebildet werden um einen Codegenerierung zu ermöglichen? 

%Welche Funktionalität kann durch die modellbasierte Entwicklung der Steuerung abgedeckt werden?

%Wie ausführlich funktioniert die Integrierung der Codegenerierung in die Software „zenon“ von COPA- DATA? 

Das Projekt Batch\_it hat sich zum Ziel gesetzt eine modellbasierte Entwicklung der Steuerung umzusetzen, wobei der erste Schritt die Konzeptionierung und der Aufbau einer Laboranlage war. Als weiteres sollte für diese eine Steuerungsapplikation mit Visualisierung mithilfe der Software zenon erstellt werden. Im letzten Schritt sollte die modellbasierte Entwicklung der Steuerung umgesetzt werden. Hierzu wurde eine Ontologie benötigt, welche die Laboranlage abbildet, und ein Aktivitätsdiagramm, welches die Prozeduren definiert. Anhand dieser Daten sollte eine Codegenerierung für die Steuerung der Anlage implementiert werden. \\\\
Die Modellierung der Chargenprozessanlage wurde mit Redundanzen versehen, um weiterführende Projekte zu unterstützen, und in einem Umfang erstellt, der die zu Verfügung gestellten Geldmittel nicht übertraf. 
%Die Modellierung der Chargenprozessanlage wurde mit Redundanzen versehen, um eine Laboranlage zu schaffen, die es ermöglicht moderne Ansätze in der Anlagen- und Verfahrenstechnik zu testen.
Die Auswahl der Bauteile war kein einfacher Prozess, da einige Parameter wie das Budget, die Kompatibilität und Verfügbarkeit beachtet werden mussten. So wurden die Rühr- und Heizelemente, die für die Reaktoren geplant waren, weggelassen, da diese zu teuere gewesen wären. Auch bei den Füllstandssensoren mussten Abstriche gemacht werden, da diese für alle Reaktoren und Tanks das Budget gesprengt hätten.
Der Aufbau der Anlage war eine langwieriger Aufgabe, die allerdings durch das Miniaturmodell beschleunigen werden konnte, da die Platzierung der Teile schon im Vorhinein überlegt werden konnte. Das Zuschneiden der Rohre musste sehr genau durchgeführt werden, da bei kleinen Abweichungen die Verbindungen schief gewesen wären. Schlussendlich konnte der Aufbau allerdings vollständig und erfolgreich abgeschlossen werden. \\\\
Die Visualisierung mithilfe von zenon konnte unproblematisch umgesetzt werden, da Vorkenntnisse mit der Software vorhanden waren. Bei dem Verbinden von SPS und zenon stand der benötigte Treiber nicht zur Verfügung. Nachdem mit dem Hersteller der SPS Kontakt aufgenommen wurde, konnte diese Problem gelöst werden. Das Einfließen lassen der Daten sowie die Steuerung über die Visualisierung funktionierte auf anhieb. Bei dem Erstellen von Phasen und Rezepte war ein umfangreiches Einlesen in die Thematik erforderlich.\\\\
Die modellbasierte Entwicklung der Steuerung wurde erfolgreich abgeschlossen. Es konnten aus der Ontologie und dem Aktivitätsdiagramm der Code für Prozeduren generiert und damit die Chargenprozessanlage gesteuert werden. Um den Code der Prozeduren in zenon zu integrieren ist allerdings eine Benutzerinteraktion benötig, weil das automatische Hinzufügen nicht möglich war. Zusätzlich gelang das Anlegen von Variablen inklusive Datentypen für die Elemente der Anlage in zenon. In zenon Logic können die Werte von Variablen mittels eines Scripts, welches beim Start ausgeführt wird, gesetzt werden. 
Die automatische Integrierung in Batch Control zum Erstellen der Rezepten wird von zenon nicht unterstützt und konnte deswegen nicht umgesetzt werden. 
Die Ontologie ist detaillierter geworden als sie die Codegenerierung benötigt hätte. Allerdings ist sie somit für Erweiterungen der Automatisierung in Zukunft vorbereitet. 


%Das Erstellen der Ontologie für Chargenprozessanlagen war eine    

%Prozeduren generierbar durch Benuttzeraktion eingespielt werden 
%Für die abgebildenten Klassen kännen Datentypen erstellt werden. 
%Für die Elemente i der Anlage können Variablen mit entsprechenden Datentypen angelegt werden.
%Die automatische INtergrierung in Batch Control (GUI) zum Erstellen der Rezepte ist nicht unterstützt. 
%In Zenon Logic können von Variablen mittels einenm Sciripts beim start, mit ihnen im Modell abgebildetten Werte initialisiert werden.















