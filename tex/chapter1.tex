% Chapter1
\chapter{Einleitung} \label{chapter:introduction}

% Inhalt
% Motiviert zum Thema und führt zum Thema hin.
% Erklärt wie man löst  
% Hintergrund und Ausgangspunkt (Heutzutage statische Systeme .. Dynamik… ) Wir wollen Methoden zur Verbesserung der Anlagen ausprobieren 
% Aufgabenstellung
% Leitfaden durch die Arbeit 		

% \section{Background and Motivation} 
% \section{Objectives of this Thesis}
% Ziel dieser Abschlussarbeit
% \section{Methodology for the Developement}  
% Methoden zur Entwicklung 
% \section{Thesis Outline}
% Abschlussarbeit Gliederung

% Mehr über die statische, hardgecodete SPS'n schreiben und die das bei Änderungen sehr viel neu  gecodet werden muss -> automatische Codegenerierung. 

% Heutzutage sind Produktionsanlagen so konstruiert, dass die Reihenfolge der einzelnen Fertigungszellen fest miteinander verkettet ist. Es herrscht eine unbewegliche, statische Folge der Stationen in dem jedes Anlagemodul autark arbeitet. Wenn es bei einem Teil zu einer Störung kommt, steht der ganze Produktionsfluss ausnahmslos. Zusätzlich ist die Flexibilität der Einsatzmöglichkeit einer Produktionsanlage eingeschränkt. Neue Module in die Verkettung hinzuzufügen erfordert enorme Umbauten und Kosten. \\\\
% Weiters kommt dazu, dass der Code auf der SPS neu programmiert werden muss, wenn ein Teil der Anlage anders verwendet werden soll. Dies hat einen Stillstand der Fabrik zur Folge. \cite{lb_SFC} Allgemein sind Produktionsanlagen sehr statisch gestaltet und können nur mit viel Aufwand geändert und angepasst werden. Bei Störungen kann nur schlecht reagiert werden was wiederum fatale Ausfälle in der Produktion nach sich zieht. \cite{wlepuschitz}





\section{Hintergrund und Ausgangspunkt}


\section{Aufgabenstellung}
% Im Rahmen des Projekts Batch_it soll eine Echtzeit- Visualisierung und Steuerung für eine existierende Anlage konzeptioniert werden. Dies umfasst das Erstellen von Phases nach IEC 61512 Standard (Funktionen wie zB. das Heizen oder Vermengen einer Flüssigkeit,...), das Entwickeln von Fehlerszenarien zur spätere Entwicklung von Fehleranalysen und die Implementierung eines schon vorhandenen Routing-Algorithmus. Als zusätzliche Vorbereitung für das Projekt BatMAS wird eine modellbasierte Codegenerierung konzeptioniert.

\section{Leitfaden durch die Arbeit}

